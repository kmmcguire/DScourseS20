\documentclass{article}
\usepackage[utf8]{inputenc}

\title{PS5 McGuire}
\author{Kevin McGuire}
\date{February 19, 2020}

\begin{document}

\maketitle

\section{Scraping}
The data I chose for scraping was primary polling data from Wikipedia. While I doubt this will be the source I will use for my project (I would prefer to having something I can reliably get for any date via API), the information itself will be used to compare the results of my analysis against. I did not use any online tutorials, as I was able to drop the new URL and selector directly into the example we did in class.

\section{API}
For the API portion, I have directly included the code I am using to collect tweets on a daily basis for my final project (stripping out my API key information). I didn't use any tutorials for setting it up beyond the documentation for the rtweet package I'm using and for Twitter's API. I also spent time on Stack Exchange learning how to fine-tune my query to exclude retweets and replies without from the initial query instead of stripping them out afterwards. I'm still working on transforming the data into usable formats that I will eventually extract meaningful information from.

\end{document}
