\documentclass [11pt]{article}

\title{2020 Democratic Primary Twitter Sentiment Analysis}
\author{Kevin McGuire}

\begin{document}
\maketitle

\section{Introduction}
This project is an exploration of how Twitter sentiment can be used to predict election results. For this project I measured Twitter sentiment of all Tweets tagging the major candidates in the 2020 Democratic presidential primary from the Iowa caucus until the race was essentially decided (clarify better when).
\section{Literature Review}
Many studies have been done on using Twitter sentiment to predict election results. Discuss them, how they are similar and different to my project.

\section{Data}
I collected all of the Tweets tagging the major candidates from the Iowa caucus to when the race was essentially decided.
\begin{itemize}
    \item Discuss major candidate criteria
    \item Discuss the different measures I extracted from the data
    \item Discuss the process of extracting the data
\end{itemize}
\section{Methods}
The data I collected was compared against both a polling average and against final results. 
\begin{itemize}
    \item Discuss how the data was collected and processed
    \item Discuss how it was compared
    \item Discuss how the final form of the predictive findings was created
\end{itemize}
\section{Findings}
The Twitter sentiment data was a good predictor of election results, and allowed for more up to date predictions close to election days than the polling average. The data was more susceptible to daily fluctuations over individual events. Discuss some examples like Bloomberg's debate debut. 
\section{Conclusion}


\nocite{*}
\bibliographystyle{IEEEannot}
\bibliography{annot}
\end{document}